The C language lends itself to writing compact and efficient code in terms of
speed and space, for a multitude of computer architectures. It is thus naturally
heavily used in the development of embedded systems. However, C's expressiveness
comes at a price: compact notation together with the language's flexible
semantics (e.g. at the level of memory management) makes it such that
non-intended behaviors can easily occur at run-time. This is due to the fact
that C programs often contain complex program statements that are syntactically
correct, but for which the semantics are difficult to understand and predict. Given that C is
often used to develop safety-critical systems, it is critical to provide
C programmers with tools that assist them in writing correct code.

In order to address these issues, the \emph{mbeddr} stack 
has been built~\cite{VoelterRKS13}. At its core, it consists of a set of
extensions to the C programming language specific to development of embedded
software. \emph{mbeddr} relies on the projectional editor provided
by JetBrains’ MPS\footnote{https://www.jetbrains.com/mps/}.
The principle of projectional editing is that code is written as a model that is
a composition of instances of concepts of a language (C or C extensions in case
of mbeddr). Compared to classical text editing, the advantages of projectional
editing are many:

\begin{itemize}
  \item The programmer can only make use of the valid and finite set of language
  constructs and their instances available at any point in the code. This is a
  step beyond classical text-based code editors in that writing syntactically
  correct code is guaranteed by construction.
  \item \emph{mbeddr} leverages the flexible language definition environment of
  MPS. This allows the definition and usage of modular extensions for the C
  language that allow the programmer to concentrate on the essential complexity of the problem
  at hand while avoiding the accidental complexity of C’s notation.
  \item Automated model transformations are used to fill in the
  details abstracted away by the extensions. This ensures that the error-prone
  code avoided by the abstractions provided by mbeddr is filled in automatically
  and correctly into the final project's compilable and runnable C code.
\end{itemize}
