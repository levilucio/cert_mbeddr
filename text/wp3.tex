Besides explicit verification, another means to increase confidence in the
correcion of the generated C code is to restrict the C language itself.
Naturally, when the subset of the language that is used is restricted, the
classes of errors that can be made by the developer are also restricted. \emph{mbeddr}
offers us the possibility of performing such restrictions by adjusting the
definition of the \emph{mbeddr} editor –- in other words, by changing \emph{mbeddr}’s C
metamodel. An example of such an adjustment would be the implementation of the
MISRA C\footnote{http://www.misra-c.com/} or the CERT
C\footnote{https://www.securecoding.cert.org/} profiles in \emph{mbeddr}’s front-end.
In practice this would consist of disallowing for example dynamic memory
allocation, data types that might have different interpretations in different
platforms and other error prone C concepts that are typically avoided in the
critical system domain.
Our goal is also to build on safe subsets of C, by proposing domain-specific
abstractions that can benefit from such safety guarantees. Targeted domains are
the automotive (e.g. by proposing \emph{mbeddr} abstractions that can simplify
construc- tion of C code for the AUTOSAR architecture) and the
aerospace.\levi{@Markus: you mentioned \emph{mbeddr} being used for satellite software
development. Are there standards or specific C abstractions for this domain that
we could think about?}

\begin{itemize}
  \item Deliverables:
  \begin{enumerate}
    \item Restriction of C to enforce a subset of the rules for both MISRA C and
    CERT C.
\item Technical proofs of the correction of the generated code for the tackled C
subsets. This will involve putting together an argument that makes use of the
results of code generation contract checkers (from Work Package 1) and the CBMC
model checker already integrated in \emph{mbeddr}.
  \end{enumerate}
  \item Effort: 4 man-months
  \item HQP: Zaur Molotnikov and Levi L\'ucio
  \item Collaborations
\end{itemize}