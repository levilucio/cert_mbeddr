\paragraph{\textbf{Objective}}: Developing infrastructure to support orthogonal
means for restricting a language according to (some) standard.\vspace{.2cm}\\
Supporting a language restriction in order to produce safer code has
demand in practice. Among such restrictions we find various standards
governing the way restrictions are to be performed. Example could be MISRA
C\footnote{http://www.misra-c.com/}, CERT Coding
Standard\footnote{https://www.securecoding.cert.org/confluence/display/seccode/SEI+CERT+Coding+Standards}
and others. Coding guidelines is another example of language restrictions 
widely spread in programming practice.

Given a language developed in mbeddr we want to be able to restrict its usage in
a particular application instance. Such restrictions might be
\begin{itemize}
  \item syntactical - the way language constructs are formulated in syntax and
  \item semantical - the way the language constructs are used, including their
  meaning in a give context
\end{itemize}

Such restrictions are often called ``rules'' in coding guidelines and standards. 

An example of a syntactical rule would be a requirement to always place a curly
brace on the next line after a loop header in C. Semantical rules can be way
more sophisticated. Examples would be: never use dynamic memory allocation, or
always check existence of a file before opening it.

Standards and coding guidelines might have other practices described
additionally to language restrictions. An example would be to
perform unit testing of each module developed. We leave such other practices out
of scope here and focus purely on a code fragment and restrictions applied
directly to it.

In this work package we propose infrastructure development for 
so-called ``language profiles''. A language profile, or simply a profile, is a
description of restrictions which have to apply to a fragment of code in a
given language. Once described in a language profile, restrictive rules might
be applied to a given language fragment insuring its compliance to the profile.

To develop profiling systems in mbeddr we will have to perform the following
\begin{itemize}
  \item Activities:
  \begin{enumerate}
    \item Analyse potential useful language restrictions for mbeddr C
    \item Come up with a number of examples for such restrictions
    \item Find commonailities and try to identify potential technical problems
    when implementing them as rules in profiles
    \item Plan a language fit to formulating rules in language profiles
    \item Develop necessary infrastructure in mbeddr to implement language
    profiles in rules as targetted by the research above. This development
    should result into the following
  \end{enumerate}
  \item Deliverables:
  \begin{enumerate}
    \item Verbal description of rules supported by profiling mechanism
    \item A language to formulate restriction rules in profiles
    \item A language to formulate profiles
    \item Checking mechanisms for a code fragment against a profile
  \end{enumerate}
  \item Effort: 4 man-months
  \item HQP: Levi L\'ucio, Zaur Molotnikov
  \item Collaborations: Markus Voelter will help with finding
  particular application domains where restrictions and/or abstractions of the C
  language are meaningful and helpful for developers.
\end{itemize}

We leave (at least for now) out of scope: compicated and hard/impossible to
ensure semantical rules (like ``a program must terminate'' rule). We do not
consider as well potential problems of the generator, which might lead to the
generated code not corresponding to the profile enforce while composing a code
fragment in MPS. These are left out for other work packages as well as for the
future work.
