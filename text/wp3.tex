\paragraph{\textbf{Objective}}: Developing infrastructure to support orthogonal
means for restricting a language according to (some) standard.\vspace{.2cm}\\
Supporting a language restriction in order to produce safer code has
demand in practice. Among such restrictions we find various standards
governing the way restrictions are to be performed. Example could be MISRA
C\footnote{http://www.misra-c.com/}, CERT Coding
Standard\footnote{https://www.securecoding.cert.org/confluence/display/seccode/SEI+CERT+Coding+Standards}
and others. Coding guidelines is another example of language restrictions 
widely spread in programming practice.

Given a language developed in mbeddr we want to be able to restrict its usage in
a particular application instance. Such restrictions are often called ``rules''
in coding guidelines and standards. An example of a (syntactical) rule would be
a requirement to always place a curly brace on the next line after a loop header
in C. Another example of a (semantical) rule would be to never allow dynamic
memory allocation, or to always check existence of a file before opening it.

In this work package we propose developing infrastructure to support
so-called ``language profiles''. A language profile, or simply a profile, is a
description of a set of restrictions which have to apply to a fragment of code
in a given language. Once described in a language profile, conformance rules
might be applied to a given language fragment insuring its compliance to the profile.
%To develop profiling systems in mbeddr we will have to perform the following
% \begin{itemize}
%   \item Activities:
%   \begin{enumerate}
%     \item Analyse potential useful language restrictions for mbeddr C
%     \item Come up with a number of examples for such restrictions
%     \item Find commonailities and try to identify potential technical problems
%     when implementing them as rules in profiles
%     \item Plan a language fit to formulating rules in language profiles
%     \item Develop necessary infrastructure in mbeddr to implement language
%     profiles in rules as targetted by the research above. This development
%     should result into the following
%   \end{enumerate}
  
\begin{itemize}
  \item Deliverables:
  \begin{enumerate}
    \item A verbal description of rules supported by a profiling mechanism and a
    language within \emph{mbeddr} to formulate language profiles.
    \item Mechanisms within \emph{mbeddr} for checking a code fragment against a
    profile.
    \item Implementation of an example profile in \emph{mbeddr} (e.g. MISRA
    C).
  \end{enumerate}
  \item Effort: 4 man-months
  \item HQP: Levi L\'ucio, Zaur Molotnikov
  \item Collaborations: Markus Voelter will help with finding
  particular application domains where restrictions and/or abstractions of the C
  language are meaningful and helpful for developers.
\end{itemize}

% \begin{itemize}
%   \item Deliverables:
%   \begin{enumerate}
%     \item\label{bul:restriction}Restriction of C to enforce a subset of the
%     rules for MISRA C.
%     \item\label{bul:abstraction} A set of domain-specific C language
%     abstractions, built on top of the C subset introduced in
%     deliverable~\ref{bul:restriction}.
% \item Technical proofs of the correction of the generated code for restrictions
% ans abstractions introduced in deliverables~\ref{bul:restriction}
% and~\ref{bul:abstraction}.
% This will involve putting together arguments that make use of the results of
% code generation contract checkers (from Work Package 1) and the CBMC model checker already integrated in \emph{mbeddr}.
%   \end{enumerate}
%   \item Effort: 4 man-months
%   \item HQP: Zaur Molotnikov, Levi L\'ucio
%   \item Collaborations: Markus Voelter will help with finding
%   particular application domains where restrictions and/or abstractions of the C
%   language are meaningful and helpful for developers.
% \end{itemize}