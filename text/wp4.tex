The technical mechanisms mentioned above are pieces in the formulation of a
complete argument for the correctness of C applications created using \emph{mbeddr}’s
front end. A complete such argument requires assembling all those pieces in a
technically and logically meaningful fashion. The result of this work implies
putting together formal and non-formal arguments that would constitute a suf-
ficiently strong proof such that the generated C code is considered certified.
Coming up with a strategy for assembling those arguments will require interac-
tion with ongoing projects at fortiss.\levi{􏰀Sebastian’s project, IETS3􏰁 and
to make use of their results}
A final point of interest of this project is to investigate how multiple C
profiles can co-exist within \emph{mbeddr}, together with their associated
certification techniques. This can be achieved by making use of \emph{mbeddr}’s native
support for variability and potentially extending it. \emph{mbeddr} allows building a
product line out of a language, by enabling or disabling certain of the
language’s concepts based on a set of features that are selected for that
language. This variability would then need to be extended to the technical
mechanisms proposed in Work Packages 1 through 3 such that each C profile can be
directly associated to a set of activities that would need to be carried out in
order to certify software built using that profile.

\begin{itemize}
  \item Deliverables:
  \begin{enumerate}
    \item A prototype for a method to integrate the artifacts built by work
    packages 1, 2 and 3 (code generation contracts, traceability, restrictions to C and
domain-specific abstractions) into a unified argument for the certi- fication of
the generated C code.
\item A prototype for a method to introduce variability in a certification argu-
ment for \emph{mbeddr}. This will be achieved by introducing variability at the level
of the technical artifacts used to build the certification argument: code
generation contracts, C restrictions and domain-specific abstrac- tions,
traceability or others that may arise during the execution of the project. The
subsets of the MISRA and CERT C profiles implemented in Work Package 3 will be
used as case studies to develop the method.
  \end{enumerate}
  \item Effort: 4 man-months
  \item HQP: Levi L\'ucio
  \item Collaborations: Members of the IETS3 project on analysis of
  requirements.
\end{itemize}