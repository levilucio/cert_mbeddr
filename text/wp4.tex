\paragraph{\textbf{Objective}}: To integrate the technical means provided by
Work Packages 1, 2 and 3 such that arguments for the certification of the C code
produced by \emph{mbeddr} can be built and reused in a modular
fashion.\vspace{.2cm}\\
The technical mechanisms mentioned above are pieces in the formulation of a
complete argument for the correctness of C applications created using
\emph{mbeddr}’s front end\markus{What is this ``front-end'' business about?
If you really need to use the word, the mbeddr *is* a front-end to C, but it
doesn't *have* one. So writing ``mbeddr's front-end'' does not make sense.}.
A complete such argument requires assembling all those pieces in a technically and logically meaningful fashion. The result of
this work implies putting together formal and non-formal arguments that would
constitute a sufficiently strong proof\markus{It's not a proof, right? It is a
``sufficiently strong argument'' or it ``provides sufficiently convincing
evidence''} such that the generated C code is considered certified.
Coming up with a strategy for assembling those arguments will require
interaction with ongoing projects at fortiss. A final point\markus{Is this
really a kinda minor subpoint of this WP?} of interest of this project is to
investigate how multiple C profiles can co-exist within \emph{mbeddr}, together with their associated certification techniques. This can
be achieved by making use of \emph{mbeddr}’s native support for variability and
potentially extending it. \emph{mbeddr} allows enabling or disabling certain
of C’s constructs for a given program, based on a set of C's features that
should be enabled or not enabled for that particular program.
This variability would then need to be extended to the technical mechanisms
proposed in Work Packages 1 through 3 such that each C profile can be directly
associated to a set of activities that would need to be carried out in order to
certify software built using that profile.

\begin{itemize}
  \item Deliverables:
  \begin{enumerate}
    \item A prototype for a method to integrate the artifacts built by work
    packages 1, 2 and 3 (code generation contracts, traceability, restrictions to C and
domain-specific abstractions) into a unified argument for the certification of
the generated C code.
\item A prototype for a method to introduce variability in a certification
argument for \emph{mbeddr}. This will be achieved by introducing variability at
the level of the technical artifacts used to build the certification argument:
code generation contracts, C restrictions and domain-specific abstractions,
traceability or others that may arise during the execution of the project. The
subset of the MISRA C profile implemented in Work Package 3 together with the
standard \emph{mbeddr} C will be used as concrete examples to develop the
method.
  \end{enumerate}
  \item Effort: 4 man-months
  \item HQP: Levi L\'ucio, Zaur Molotnikov
  \item Collaborations: Sebastian Voss and the members of the IETS3 project on
  analysis of requirements will provide their expertise on the formulation of
  arguments of correctness for software.
\end{itemize}