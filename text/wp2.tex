Traceability is a fundamental artifact to enable certification as it allows
knowing which requirements are implemented in the final software product and how
that implementation was done. In the context of \emph{mbeddr}, traceability links all
artifacts used throughout the development of embedded C software, starting from
a set of requirements and finishing with the generated code. Although
traceability between artifacts (textual requirements, \emph{mbeddr} code and C code) is
natively supported by \emph{mbeddr}, the traceability between \emph{mbeddr} specifications and
the generated C code connects abstract syntax to lines of code. \emph{mbeddr}
developers have expressed the desire to have more advanced and informative
traceability built in terms of how an abstract syntax tree of an \emph{mbeddr}
specification is related to the abstract syntax tree of the generated C code.
The introduction of model transformation languages in \emph{mbeddr}, as proposed in
Work Package 1, provides a technical base to tackle this problem.

\markus{mbeddr's existing trafo lang does this.}

\begin{itemize}
  \item Deliverables:
  \begin{enumerate}
    \item Prototype in \emph{mbeddr} for building and storing explicit traceability
    from the embedded application requirements, to the corresponding \emph{mbeddr}
    specification abstract syntax, to the corresponding abstract syntax tree for
    the generated C code.
  \end{enumerate}
  \item Effort: 2 man-months
  \item HQP: Levi L\'ucio
\end{itemize}
