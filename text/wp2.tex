\paragraph{\textbf{Objective}}: To establish explicit traceability mechanisms
for a C application developed in \emph{mbeddr}, from the requirements to
the generated code.\vspace{.2cm}\\
Traceability is a fundamental artifact to enable certification as it allows
knowing which requirements are implemented in the final software product and how
that implementation was done. In the context of \emph{mbeddr}, traceability links all
artifacts used throughout the development of embedded C software, starting from
a set of requirements and finishing with the generated code. 

Given that a new model transformation language for code generation will be
introduced in \emph{mbeddr} by Work Package~\ref{sec:wp1}, current
traceability mechanisms within \emph{mbeddr} will have to be reviewed and
adapted. In particular, it will be necessary to build and store traces between
the abstract syntax element of the \emph{mbeddr} language and of the \emph{C}
language.
%\markus{mbeddr's existing trafo lang does this.}

\begin{itemize}
  \item Deliverables:
  \begin{enumerate}
    \item Prototype in \emph{mbeddr} for building and storing explicit traceability
    from the embedded application requirements, to the corresponding \emph{mbeddr}
    specification abstract syntax, to the corresponding abstract syntax tree for
    the generated C code.
  \end{enumerate}
  \item Effort: 2 man-months
  \item HQP: Levi L\'ucio, Zaur Molotnikov, Hiwis student
  \item Collaborations: Members of the IETS3 project.
\end{itemize}
