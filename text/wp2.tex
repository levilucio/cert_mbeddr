\paragraph{\textbf{Objective}}: To automatically provide information on
traceability between the requirements for an \emph{mbeddr} specification, the \emph{mbeddr}
specification itself and the generated C code.\vspace{.2cm}\\
Traceability is a fundamental artifact to enable certification: it allows
knowing which requirements are implemented in the final software product and how
that implementation was done. In the context of \emph{mbeddr}, traceability links all
artifacts used throughout the development of embedded C software, starting from
a set of requirements and finishing with the generated code. 

This work package will concentrate on replacing the current traceability
management mechanism between an \emph{mbeddr} specification and the
automatically generated C code. This is necessary due to
the fact that the new model-to-model transformation language in Work Package 1
introduces its own traceability management mechanism.

%  a new model transformation language for code generation will be
% introduced in \emph{mbeddr} by Work Package~\ref{sec:wp1}, MPS-based
% traceability between the abstract syntax elements of an \emph{mbeddr}
% specification and the abstract syntax elements of the generated C code will need
%  have to be created, stored and managed.
% such that this information can be used for certification activities.
% \markus{Sorry to be a PITA, but this formulation really does not make sense.
% There are two kinds of traceability: those traces entered manually by users
% between model model elements (eg from code to requirements). THis is already
% done by mbeddr, and there really is nothing to do. It's done. The other one is
% the automatic trace used by MPS internally to track how, during
% transformations, nodes are transformed into other nodes. This is also done, but
% only for MPS' existing trafo languages. Now, you have two options for what you
% can do: 1) you make sure that your new trafo language also creates these
% MPS-based traces during trafo. This is important for eg debugging. 2) you may
% want to create reports on that data that can be used during certification.}

\begin{itemize}
  \item Deliverables:
  \begin{enumerate}
    \item Prototype within \emph{mbeddr} for storing within the MPS
    framework the traceability information produced by the model-to-model
    transformation language introduced in Work Package 1.
    \item Prototype within \emph{mbeddr} for creating reports on traceability
    from requirements to the generated C code. These reports are to be used as
    part of the certification process.
  \end{enumerate}
  \item Effort: 2 man-months
  \item HQP: Levi L\'ucio, Zaur Molotnikov, Hiwis student
  \item Collaborations: Members of the IETS3 project.
\end{itemize}
