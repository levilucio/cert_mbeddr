During project implementation we count on having early feedback from our
industrial partners itemis, OHB Systems (satellite onboard software developer),
Siemens. This feedback will allow us to refine our prototypes throughout
several iterations.

Dissemination will be done primarily at the industrial level by targeting users
of the version of \emph{mbeddr} built by our partners at itemis AG. Given the
mechanisms we propose in the project can be used at the level of the MPS IDE,
the MPS community will also be targeted for dissemination. Foreseen
dissemination actions include writing a project website, user guides, online
tutorials and building short videos that can integrate both the \emph{mbeddr} and MPS
websites. Other actions include organising workshops together with members of
the \emph{mbeddr} or MPS communities as well as hands-on tutorials.
Dissemination and community building within fortiss itself will done also using
hands-on tutorials where users will be taught about how to build certifiable C
code.􏰁 Additional dissemination actions will be done by writing joint papers,
attending specialized conferences and proposing or joining workshops in the
domain in order to build academic interest in the \emph{fortiss-mbeddr} tool as
a complete means for building certified C code.

\begin{itemize}
  \item Deliverables:
  \begin{itemize}
    \item{\textbf{D5.1}:} Project website, user guide, online tutorial, screen
    casts
	\item{\textbf{D5.2}:} Hands-on tutorial
	\item Academic papers accepted at relevant workshops, conferences and journals
	(venues to be determined)
  \end{itemize}
  \item Effort: 17 man-months
  \item HQP: Levi L\'ucio, staff researcher, M.Sc. student, HiWi student 
  \item Collaborations: all collaborators.
\end{itemize}