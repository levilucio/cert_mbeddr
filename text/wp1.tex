% The \emph{mbeddr} editor helps the developer writing correct C code. On the one
% hand, \emph{mbeddr}’s front-end helps the developer by providing editing
% assistance and good abstractions. On the other other hand, the generated code
% can be verified using the CBMC\cite{KroeningT14} C model checker, there are no guarantees that the
% semantics of the \emph{mbeddr} C specification is kept in the generated
% code
% 
% \markus{I don't quite understand why you mention CBMC here.
% Doesn't it provide (some of the) assertions you are aiming for? I would omit
% this here, and in a separate paragraph explain, how and why unit testing and
% even CBMC are not good enough.}.

%  Currently, the verification infrastructure
% within \emph{mbeddr} includes: a framework for unit-testing; the
% CBMC\cite{KroeningT14} C model checker for analysing the generated C code.
% Unfortunately these techniques alone or even together are not sufficient to
% make sure that the generated C code is correct, as none of them explicitly
% targets the relation betweem

\paragraph{\textbf{Objective}}: To provide the technical means within
\emph{mbeddr} that allow verifying the generated C code.\vspace{.2cm}\\
One of the main bottlenecks when trying to provide an argument for the
correctness of the C code generated from \emph{mbeddr} lies in the fact that the
code generators that transform \emph{mbeddr} C specifications into real C code
cannot be explicitly verified. This is a crucial problem for the adoption of
\emph{mbeddr} in practice: a case needs to be made for the validity of the
generated C code. Making this case will encourage C developers to use the tool;
its omission may prevent \emph{mbeddr} from being used for the
development of critical systems, one of the main domains the tool is aimed at.
Verifying code generation falls in the domain of the analysis of model
transformations and can be tackled using the approach described
in~\cite{Lucio2014,Oakes2015}.

Note that because \emph{mbeddr} is built with MPS, work on verifying code
generation within \emph{mbeddr} can be used for other code generators within
the MPS framework itself. This will allow us to reach a broader audience for our
results and to open perspectives for more industrial collaborations and deeper scientific insights.
% \levi{What is the size of the MPS
% community? Would this be interesting for that community?}\markus{Yes,there are other users (Daimler,
% BOSCH), who could be interested in this.}
% \markus{Deliverable IMHO: 1 a new M2M trafo lang for MPS 2 a contract prover for
% this lang 3 reimplementation of some of mbeddr's trafos with this new language,
% 4 plus correctness proofs for these trafos. Me and the team here would help you
% integrate 1 and 2 into MPS. And the help design and implement 3. You would help us do 4.}
\begin{itemize}
  \item Deliverables:
  \begin{enumerate}
    \item A prototype within \emph{mbeddr} of a model-to-model transformation
    language to replace current model-to-text code generators.
	\item A prototype within \emph{mbeddr} of a pre- / post-condition contract
	prover using the new model-to-model transformation language.
\item Reimplementation of a significant sample of \emph{mbeddr}'s code
generators for the new model-to-model transformation language.
	\item A technical proof of the correctness of the implemented code generators.
  \end{enumerate}
  \item Effort: 6 man-months
  \item HQP: Levi L\'ucio
  \item Collaborations: Markus Voelter and the \emph{mbeddr} team at itemis (and
  potentially JetBRAINS) will help Levi L\'ucio in integrating the new
  transformation language within MPS and with the design the new \emph{mbeddr}
  code generators. Cl\'audio Gomes and Bentley Oakes will provide expertise on
  model transformation languages and their analysis. A collaboration between Levi
   L\'ucio and Markus Voelter related to this work
  package has been ongoing informally during most of the year of 2015.
\end{itemize}